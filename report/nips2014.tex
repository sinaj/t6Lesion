\documentclass{article} % For LaTeX2e
\usepackage{nips14submit_e,times}
\usepackage{hyperref}
\usepackage{url}
%\documentstyle[nips14submit_09,times,art10]{article} % For LaTeX 2.09
\usepackage{graphicx}              % to include figures
\usepackage{epstopdf}
\usepackage{amsmath}               % great math stuff
\usepackage{amsfonts}              % for blackboard bold, etc
\usepackage{amsthm}                % better theorem environments

\usepackage{notoccite}
\usepackage[numbers,sort&compress]{natbib}



\title{Multiple Sclerosis Lesion Segmentation in MRI Images }

\author{
Gautham Vasan \\
Department of Computing Science\\
University of Alberta\\
Edmonton, T6E2H2 \\
\texttt{vasan@ualberta.ca} \\
\And
Jared Rewerts \\
Department of Computer Engineering \\
University of Alberta \\Edmonton, T6E2H2 \\
\texttt{rewerts@ualberta.ca} \\
\And
Megha Panda \\
Department of Computing Science\\
University of Alberta \\Edmonton, T6E2H2 \\
\texttt{meghaanu@ualberta.ca} \\
\texttt{email} \\
\And
Parisa Mohebbi \\
Department of Computing Science \\
University of Alberta \\Edmonton, T6E2H2 \\
\texttt{mohebbi@ualberta.ca} \\
\And
Reza Sobhannejad \\
Department of Computing Science \\
University of Alberta \\Edmonton, T6E2H2 \\
\texttt{sobhanne@ualberta.ca} \\
\And
Sina Jalali  \\
Department of Computing Science \\
University of Alberta \\Edmonton, T6E2H2 \\
\texttt{jalali1@ualberta.ca} \\
}

% The \author macro works with any number of authors. There are two commands
% used to separate the names and addresses of multiple authors: \And and \AND.
%
% Using \And between authors leaves it to \LaTeX{} to determine where to break
% the lines. Using \AND forces a linebreak at that point. So, if \LaTeX{}
% puts 3 of 4 authors names on the first line, and the last on the second
% line, try using \AND instead of \And before the third author name.

\newcommand{\fix}{\marginpar{FIX}}
\newcommand{\new}{\marginpar{NEW}}

\nipsfinalcopy % Uncomment for camera-ready version

\begin{document}


\maketitle
% Abstract
\begin{abstract}
The abstract paragraph should be indented 1/2~inch (3~picas) on both left and
right-hand margins. Use 10~point type, with a vertical spacing of 11~points.
The word \textbf{Abstract} must be centered, bold, and in point size 12. Two
line spaces precede the abstract. The abstract must be limited to one
paragraph.
\end{abstract}

%---------------------------------------- Introduction-------------------------------------------%
\section{Introduction}
What is MS? Why it is an interesting problem? Problem definition and motivation. What are the challenges?

\subsection{Related Work}
Segmentation of Lesions in MRI images is an active area of research. The 3 main objectives 1) Extracting features that can differentiate healthy tissues from scar tissues. 2) Selecting the most relevant features that help achieve the task 3) Improving the performance of the classification.

\subsubsection{Review of Segmentation Methods}
Here we talk about what methods have been used so far. The features and the classifier used should be discussed. Basically a literature survey. We could just use Review paper that Dana sent us earlier. 

\subsubsection{ Exploring Combinations }
What is it that we intend to do, i.e., compare performance of different classifiers and features.

%------------------------------------------- Features ----------------------------------------------%
\section{Feature Detection}
%What is feature extraction? Why is it important? Reasons for using the particular features
Image analysis aims at reducing information to a subset that is relevant to the task in hand. Information reduction often happens gradually with information being reduced until the desired result is extracted from the data. \cite{Toennies} The first level of reduction computes local features that
are assumed to pertain to objects of interest. 
\subsection{Context (Haar-Like Features)}

\subsection{Image Filters}

\subsubsection{Leung-Malik(LM) Filter Bank}

\subsection{Entropy \& Gaussian based Features)}

\subsection{Atlas Features}

All the features need to explained briefly with good images.

%------------------------------------------------ Classifiers ------------------------------------------%
\section{Classifiers}

Reasons for using each classification method. Brief description about them. No need for images here I guess.  

\subsection{Support Vector Machines}

\subsection{Neural Networks}

\subsection{k-Nearest Neighbours}

\subsection{Random Forests}

\subsection{Markov Random Fields}

\subsection{Logistic Regression}

%-----------------------------------Experiment Design -------------------------------------------------------%
\section{Experiment Design}
Pipeline Diagram is required here

\subsection{Validation Measures}
False Positive, false negative, etc. Why Dice is preferred?
\subsubsection{Dice Score}
\subsubsection{Accuracy}
\subsubsection{Sensitivity}
\subsubsection{Detections}

\subsection{Training and Test Data}
BrainWeb and Miccai Challenge Data. Explain how sampling is done on data to train the classifier. 

% --------------------------------------- Results ----------------------------------------------------%
\section{Results}
Lots of Images! Table with comparative results. Explanation for why we get these results. 

\subsection{State of the art results}
Brief description of the benchmark

\subsection{Discussion}
Detailed analysis of the results we get from different results

%-------------------------------------- Conclusion ------------------------------------------%
\section{Conclusion}

% ---------------------------------------- Acknowledgements ----------------------------------%
\section{Acknowledgements}

%------------------------------------------- References ----------------------------------------%
\section{References}

\bibliography{bibliography}
\bibliographystyle{plain}

\end{document}
